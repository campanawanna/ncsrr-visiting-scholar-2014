\documentclass[11pt]{article}

% enables Arial font
\usepackage{helvet}
\renewcommand{\familydefault}{\sfdefault}

\title{Identification of human locomotion control based on perturbed walking
and running data under the constraints of a biologically actutuated plant.}

\author{Jason K. Moore\\
  Postdoctoral Research Associate\\
  Parker Hannifin Human Motion \& Control Laboratory\\
  Cleveland State University
}

\begin{document}

\maketitle

\section{Introduction}

The current research focus at the Human Motion \& Control Laboratory in
Cleveland is focused on the identification and application of biolligically
inspired controllers to powered prothestics. My current role is based around
using data driven approaches to identify controllers capable of reproducing
natural human gait patterns from data collected from able bodied people in
walking and running. The current identified controllers are structured to map
to the sensors and acutuators available on non-nuerologically connected
prosthetics.

There are two approaches we are working with:

1. Direct identification given measurements of the controllers inputs and
outputs.
2. Indirect identifcation given the measuremetns of the system's kinematics and
a known plant model.

Both of these methods. The direct identification requires a noise model for the
controller and a persistnetly excited input to reduce bias in the estimates of
the model. The indirect method is attractive because any identification method
can be applied.

We are in need a sufficiently complex, but simple plant model that and
similarly a simple control model set that includes the ``true'' model. There
are many models that can be used, but a particularly interesting, sucessful.
and relevant platn model is the one presented by \cite{Wang2012}. Wang et. al,
demonstrated 

%Specific Aims: Review the rationale for your study, and state your hypotheses
%and specific aims.
%
%Methods: Provide an overview of the study design and methods. Clearly identify
%why an on-site collaboration would be beneficial to the proposed work.
%Potential bottlenecks should also be described with proposed solutions.
%
%Expected Results: State the expected outcome of your work and its impact on the
%field.
%
%Relevance to Rehabilitation: Articulate the importance of this project to
%rehabilitation.
%
%Contributions to the Biomechanical Simulation Community:  Describe the
%software, data, and/or models that will be made available to the biomechanics
%community at the end of the visit.
%
%Suitability of Applicant:  Explain why your background is appropriate for the
%proposed research problem. Describe your prior experience using OpenSim or
%other modeling and simulation approaches.

\end{document}
