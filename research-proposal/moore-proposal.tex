\documentclass[11pt]{article}

% enables Arial font
\usepackage{helvet}
\renewcommand{\familydefault}{\sfdefault}

\usepackage[top=0.75in, bottom=0.75in, left=1.0in, right=1.0in]{geometry}

\title{Identification of human locomotion control with perturbed walking and
running data under the constraints of a biologically actutuated plant}

\author{Jason K. Moore\\
  Postdoctoral Research Associate\\
  Parker Hannifin Human Motion \& Control Laboratory\\
  Cleveland State University
}

% Expectations from the reviewers:

%Specific Aims: Review the rationale for your study, and state your hypotheses
%and specific aims.
%
%Methods: Provide an overview of the study design and methods. Clearly identify
%why an on-site collaboration would be beneficial to the proposed work.
%Potential bottlenecks should also be described with proposed solutions.
%
%Expected Results: State the expected outcome of your work and its impact on the
%field.
%
%Relevance to Rehabilitation: Articulate the importance of this project to
%rehabilitation.
%
%Contributions to the Biomechanical Simulation Community:  Describe the
%software, data, and/or models that will be made available to the biomechanics
%community at the end of the visit.
%
%Suitability of Applicant:  Explain why your background is appropriate for the
%proposed research problem. Describe your prior experience using OpenSim or
%other modeling and simulation approaches.

\begin{document}

\maketitle

\section*{The Gist}

During the 2014 NCSRR visiting scholar program, I intend to extend the optimal
control work presented in the paper ``Optimizing Locomotion Controllers Using
Biologically-Based Actuators and Objectives'' by Wang et. al~\cite{Wang2012} in
such a way that the controller parameters are identified by minimizing the
difference in the simulation's kinematics and the measured kinematics from a
large set of data collected from a sample of subjects walking and running while
under the influence of lateral and longitudinal random perturbations. The
resulting identified controller is hypothesized to be more realistic than any
other existing model.

\section*{The Hypothesis}

My hypothesis is that it is possible to converge on more realistic controllers
than predictive control methods have shown by utilizing an indirect
identification approach with a large set data collected from walkers and
runners who have been excited by random perturbations in combination with a
sufficiently realistic plant model.

Also, with careful attention to model complexity, simulation speed, and
optimization methods I believe we can increase the speed of the computations
such that simulating for 1 to 10 time minutes is computationally tractable
using modern cloud computing resources and iteration times can be similar to or
less than those presented in \cite{Wang2012}.

\section*{The Challenges}

Reducing computational speed while retaining model fidelity and accuracy.

\section*{Introduction}

The current research focus at the Human Motion \& Control Laboratory in
Cleveland is focused on the identification and application of biologically
inspired controllers to powered prosthetics. My current role is based around
using data driven approaches to identify controllers capable of reproducing
natural human gait patterns from data collected from able bodied people in
walking and running. The current identified controllers are structured to map
to the sensors and actuators available on non-neurologically connected
prosthetics.

There are two approaches we are working with:

\begin{itemize}
  \item Direct identification given measurements of the controllers inputs and
    outputs.
  \item Indirect identifcation given the measuremetns of the system's
    kinematics and a known plant model.
\end{itemize}

The direct identification methods is attractive because it doesn't requier a
model of the plant. But it is limited due to the inability to acutally measure
the controller inputs and outputs directly and that it requires a suffciently
adequate noise model for the controller and along with persistent excitation to
reduce bias in the estimates of the model.

The indirect method has been becoming more attractive to us attractive because
any identification method can be applied with no effect from bias, but the
downside is that the success is depdendent on computationally difficult
optimization problem.

We are in need a sufficiently complex, but simple plant model that and
similarly a simple control model set that includes the ``true'' model. There
are many models that can be used, but a particularly interesting, sucessful.
and relevant platn model is the one presented by \cite{Wang2012}. Wang et. al,
demonstrated 

\section*{How}

The project can be organized into these topics:

\begin{itemize}
  \item Collect and process perturbed gait data at CSU before the visiting
    scholarship.
  \item Develop an efficient plant model based on \cite{Wang2012} in Opensim.
  \item Develop an identification/optimization routine based on kinematic
    tracking, minimal metabolic effort, and joint torques.
  \item Deploy the software on cloud architecture for maximum compute power and
    minimum speed.
\end{itemize}

\subsection*{Human Subject Data}

At CSU, we have a modern gait lab that includes a Motek Medical hardware and
software suite. The suite includes an Opsrey Digital RealTime Motion Capture
system from Motion Analysis and a R-Mill Treadmill from ForceLink. The
treadmill has dual belt and force plates and it includes the ability to actuate
the treadmill surface laterally and in pitch. The belts can accelerate up to
$20 m/s^{-2}$, the lateral motion of 0.1 m/s, and a pitching motion of 20
degrees/s. We have developed a protocol that applies random longitudinal and
lateral perturbations over four minute trials sampled at 100 hz. This ability
to actuate the treadmill surface is critical to obtain data rich enough for
controller identification.

% TODO : Show example perturbed data.
% TODO : Figure out what speeds we are actually using.

We are currently collecting data from able-bodied subjects while walking and
running at speeds from 0.8 m/s to 4.0 m/s while under the influence of various
white noise based perturbations. I will have sets of rich motion capture,
inertially compensated ground reaction forces data along with estimates of
joint kinematics, torques, muscle force and length data available at the time
of the visiting scholarship from this system and from sufficient sample of
able-bodied subjects.

\subsection*{Plant Model}

There are a variety of plant models capable of walking and running that
available for use, potentially including Wang's Open Dynamics Engine based
model, but my visit will focus on the development of a simple but sufficiently
complex plant model using Opensim that capable of simulation speeds much faster
than real time and of sufficient numerical accuracy. \footnote{I've been in
  contact with the authors of \cite{Wang2012} and understand that there has
been work on a Simbody based model. I also asked Wang if the source code for the
2012 paper could be shared, but he seemed reluctant.}

Wang's model is a 3D, 30 degree of freedom model which can be simulated at real
time speeds on standard personal hardware. His plant includes simple contact
for the heel and ball of the feet and Hill type muscle models to actuate the
joints most important for gait analysis and simple joint torque actuation for
non-gait-critical joints.

In addition add we will add the ability to specifics the longitudinal and
lateral acceleration of the ground plane as a specified input. This will allow
use of the perturbation data.

\subsection*{Controller Model}

Wang et al. demonstrated that the set of control models based on a control
architecture with positive muscle force and length feedback paired with typical
proportional-derivative control on the joint angles is a very likely to contain
the ``true'' controller used in human walking and running. This is indicated by
the good prediction of joint torques and more natural looking motion. I will
start by implementing Wang's controller model and making use of his results as
initial guesses for the controller parameters.

We also have some controller architectures based on gait phase scheduling that
we've been exploring with direct identification methods that can be considered
too.

\subsection*{Identification}

\cite{Wang2012} approaches the ``identification'' of a plausible controller
from the predictive stand point. A plant and controller model structure is
selected and the controller parameters are identified by minimizing given a
cost function based only on minimizing metabolic effort, minimizing non-gait
critical joint torques, and keeping several priority tasks, such as body
orientation and forward speed. I'd like to modify this concept by removing the
task based costs and replacing them with the minimization of the difference in
the marker locations on the model and measurements collected from walkers and
runners while being perturbed.

Wang has shown that his controller is able to produce visually motions that
also exhibit realistic joint torques, at least in walking. I hypothesize that
more realistic controller parameters can be identified by tracking kinematic
data from walkers and runners that are being perturbed with known (measured)
perturbations. We will minimize a cost function of this form

\begin{equation}
  \mathbf{y}\mathbf{bf}\mathbf{y}^T +
  \mathbf{u}\mathbf{R}\mathbf{u}^T dt
\end{equation}

where $\mathbf{y}$ is a vector of the difference in the measured and simulated
marker coordinates and $\mathbf{u}$ is a vector of the average rate of
metabolic expenditure of the muscles and the sum of the square of the joint
torques which will be identical to \cite{Wang2012}'s definition.

This formulation will require simulating the system from 1 to 10 minutes of
real time and ensuring that the difference in the states and the experimental
measurements (sampled at 100 hz) are minimal. Wang's model simulated for 10s at
2400 Hz for accuracy resulting in 10 hour computations on a 50 core cluster to
converge on a solution. This means that using Wang's work as is, the
computations could take 60 to 600 hours for the trajectory tracking.

For time optimal convergence times in the identification I plan to explore some
or all of the following:

\begin{description}
  \item[Reduced order plant] It may not be necessary to employ a 30 DoF model
    for antedate model fidelity. If needed the model can likely even be reduced
    to a planar 2D model and still provide realistic results. Furthermore, the
    plant model could be reduced to a simple walker (i.e. no knee, ankle, or
    upper body degrees of freedom) if we place physical restrictions on the
    human subjects degrees of motion.
  \item[ODE Integration] Simbody includes several ODE integration routines that
    include explicit and implicit solvers. CVODES may be able to be sped up by
    providing an analytical Jacobian for the implicit solver or parallelizing the
    numerical Jacobian computation. We may also be able to use or create a 2nd
    order implicit Euler method that can compute quickly regardless of the
    stiffness due to ground contact.
  \item[Direct Collocation] Wang uses an evolutionary derivative-free
    algorithm that requires shooting. It is also possible to use a strictly
    implicit formulation of the system paired with direct collocation for
    potentially faster convergence times even though the number of free
    parameters is larger.
  \item[Cloud Computing] The computations will be parallelized as much as
    possible and run using Amazon's Web Service cloud computing services (~108
    cores @ \$2.4 per hour).
\end{description}

\section*{Relevance to Rehabilitation}

The results of these studies will be directly used at the CSU Human Motion \&
Control Lab to improve the controllers in power prosthetics and exo-skeletons.
These devices will and are already playing an important role in revalidation.

But, in general, realistic control models for walking and running will enhance
a broad range of research. As much as realistic musculo-skeletal models have
helped, control models may do the same.

\section*{Contributions}

All of the products developed during, prior, and after the visiting scholarship
that are related to this proposal will be made available under permissible
licenses (BSD/MIT like and/or CC-BY-4.0) and shared via paper, code, and data
repositories on the world wide web. In particular, these are the likely
products that will be valuable to the community:

\begin{description}
  \item[Plant Models] The source code and/or any Opensim xml descriptions for
    all of the plant models will be released to the SimTK community and be
    hosted on Github.
  \item[Source Code] All of the source code (plant and controller models,
    optimization routines, cloud infrastructure provisioning, etc) will be
    shared via Github and structured such that all published results are
    computationally reproducible.
  \item[Data] The anonymized human subject data will be shared via Figshare
    and/or institutional repositories (CSU or Stanford) and released under the
    CC-BY-4.0 License.
  \item[Papers] At least one paper will be published in an open access journal
    under the CC-BY-4.0 licenses along with supplementary material (source code
    and data) to reproduce the results. A preconfigured virtual machine that
    can run the computations on a cloud infastructure will accompany the
    results.
\end{description}

\section*{My Background}

I am adept at mathematical modeling and simulation and have extensive
experience with both vehicle and biomechanical systems. I was initially trained
at UC Davis and specialized in Sports Biomechanics for my graduate work where
modeling, simulation, system identification played an important role in my
graduate work. I co-develop and run a project, PyDy, that provides an open
source modeling and simulation framework for complex multi-body systems. My
Github profile provides a ``resume'' of sorts showing my open source
contributions to scientific computing in vehicle dynamics and biomechanics.

I am a beginner at Opensim and Simbody due to the fact the tools were not used
in the laboratories that I've worked at thus far. But I've spent some time
this year working through the tutorials available for Simbody and Opensim and
am lightly proficient.

I've tended to develop my own models with PyDy in the past. But I've spent
time working on example walking models

\bibliographystyle{plain}
\bibliography{references}

\end{document}
