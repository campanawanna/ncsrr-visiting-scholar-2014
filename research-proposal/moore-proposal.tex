\documentclass[11pt]{article}

% enables Arial font
\usepackage{helvet}
\renewcommand{\familydefault}{\sfdefault}

\usepackage[top=0.75in, bottom=0.75in, left=1.0in, right=1.0in]{geometry}

\usepackage{hyperref}
\usepackage{xcolor}
\usepackage{xcolor}
\usepackage[normalem]{ulem}
\usepackage{hyperref}

\hypersetup{colorlinks=true,
            pdfborderstyle={/S/U/W 1},
            pdfborder=0 0 1,
            linkcolor=blue,
            citecolor=blue,
            urlcolor=blue}

\title{Identification of human locomotion control with perturbed walking and
running data under the constraints of a biologically actuated plant}

\author{Jason K. Moore\\
  Postdoctoral Research Associate\\
  Parker Hannifin Human Motion \& Control Laboratory\\
  Cleveland State University
}

% Expectations from the reviewers:

%Specific Aims: Review the rationale for your study, and state your hypotheses
%and specific aims.
%
%Methods: Provide an overview of the study design and methods. Clearly identify
%why an on-site collaboration would be beneficial to the proposed work.
%Potential bottlenecks should also be described with proposed solutions.
%
%Expected Results: State the expected outcome of your work and its impact on the
%field.
%
%Relevance to Rehabilitation: Articulate the importance of this project to
%rehabilitation.
%
%Contributions to the Biomechanical Simulation Community:  Describe the
%software, data, and/or models that will be made available to the biomechanics
%community at the end of the visit.
%
%Suitability of Applicant:  Explain why your background is appropriate for the
%proposed research problem. Describe your prior experience using OpenSim or
%other modeling and simulation approaches.

\begin{document}

\maketitle

\section*{Specific Aims}

The current research at the \href{http://hmc.csuohio.edu}{Human Motion \&
Control Laboratory} in Cleveland is focused on the identification and
application of biologically inspired controllers to powered prosthetics. My
work is are based around using data driven approaches to identify feedback
controllers capable of reproducing natural walking and running. We have been
working with controllers structured to connect the sensors and actuators
available on non-neurologically connected prosthetics.

There are two approaches we are working with: (1) direct identification given
measurements or estimates of the controllers' inputs and outputs and (2)
indirect identification given the measurements of the system's kinematics and a
``known'' plant model.

The direct identification method is attractive because it does not require a
model of the plant. But it is limited due to the inability to measure the
controller inputs and outputs directly and in that it requires a sufficiently
adequate controller noise model in addition to sufficient and persistent
external excitation to reduce bias in the estimates of the controller.

The indirect method has been become more appealing to us because any
identification method can be applied with no effect from bias, but the downside
is that the computational difficulty is much higher. This proposal details how
we can use the expertise available at the National Center for Simulation
Rehabilitation Research (NCSRR) to help make indirect identification methods
for gait control attainable.

During the 2014 NCSRR visiting scholar program, I intend to extend the optimal
control work presented in the paper ``Optimizing Locomotion Controllers Using
Biologically-Based Actuators and Objectives'' by Wang et. al~\cite{Wang2012} in
such a way that the controller parameters are identified by minimizing the
difference in the simulation's kinematics and the measured kinematics from a
large set of data. The data will be collected from a sample of subjects walking
and running while under the influence of random lateral and longitudinal
perturbations. I hypothesize that the resulting identified controller will be
more realistic than any other existing models.

Also, with careful attention to model complexity, simulation speed, and
optimization methods I believe we can increase the speed of the computations
such that simulations of 1 to 10 minutes are computationally tractable using
modern cloud computing resources. Iteration times will likely be similar to or
less than those presented in \cite{Wang2012}.

\section*{How}

The project can be broken into four main topics:

\subsection*{Human Subject Data}

At CSU, we have a modern gait lab that includes a
\href{http://www.motekmedical.com}{Motek Medical} hardware and software suite.
The suite includes an Osprey Digital RealTime Motion Capture system from
\href{http://www.motionanalysis.com}{Motion Analysis} and a R-Mill Treadmill
from \href{http://www.forcelink.nl}{ForceLink}. The treadmill has dual belt and
dual force plates with the ability to actuate the treadmill surface laterally
and in pitch. The belts can accelerate up to $15 m/s^{-2}$, the lateral speed
can reach 0.1 m/s, and the pitching rate can reach 20 degrees/s. We have
developed a protocol that applies random longitudinal and lateral perturbations
over four minute trials sampled at 100 hz. This ability to actuate the
treadmill surface is critical to obtain data rich enough for controller
identification.

% TODO : Figure out the acceleration limits for the sway and pitch. These are
% not listed in the R-Mill's specifications.

We are currently collecting data from able-bodied subjects while walking and
running at speeds from 0.8 m/s to 4.0 m/s while under the influence of various
white noise based perturbations. I will have data sets of rich motion capture
and inertially compensated ground reaction forces along with estimates of joint
kinematics, torques, muscle force, and muscle length data from sufficient a
sample of subjects available by the time of the visiting scholarship.

\subsection*{Plant Model}

There are a variety of plant models capable of walking and running available
for use, potentially including Wang et. al's Open Dynamics Engine based model,
but my visit will focus on the development of a simple but sufficiently complex
plant model using Opensim that is capable of simulation speeds much faster than
real time and of sufficient numerical accuracy.~\footnote{I've been in contact
  with the authors of \cite{Wang2012} and understand that there has been work
  on a Simbody based model. Dr. Jack Wang has also expressed interest in
collaboration.}

Wang et. al's model is 3D and has 30 degree of freedom which can be simulated
at real time speeds on standard hardware. His plant includes simple contact for
the heel and ball of the feet with Hill type muscle models to actuate the
joints most important for gait along with simple joint torque actuation for
non-gait-critical joints.

My contact model will compute friction force as a function of the velocity
difference between the surface and a point on the foot. The plant will have the
additional ability to specify the velocity of the ground plane as a specified
time dependent input. This will allow use of the measurements of the
perturbations in simulation.

\subsection*{Controller Model}

Wang et al. demonstrated that the set of control models based on a control
architecture with positive muscle force and length feedback paired with typical
proportional-derivative control on the joint angles is a very likely to contain
the ``true'' controller used in human walking and running. This is indicated by
the good prediction of joint torques and more natural looking motion. I will
start by implementing Wang's controller model and making use of his results as
initial guesses for the controller parameters.

% TODO : Maybe remove the following sentence.

We also have some controller architectures based on gait phase scheduling that
we've been exploring with direct identification methods that can be considered
too.

\subsection*{Identification}

Wang et. al  approach the ``identification'' of a plausible controller from the
predictive stand point. A plant and controller model structure is selected and
the controller parameters are identified by minimizing given a cost function
based only on metabolic effort, non-gait critical joint torques, and keeping
several priority tasks, such as body orientation and forward speed. I would
like to modify this concept by removing the task based costs and replacing them
with the minimization of the difference in the marker locations on the model
and measurements collected from walkers and runners while being perturbed.

Wang et. al has shown that his controller is able to produce motions that also
exhibit realistic joint torques, at least in walking. I expect our method to
find more realistic predictions of joint torques due to the difference
constraints we impose. For example, we will minimize a cost function of this form

\begin{equation}
  R = \sum \mathbf{y}\mathbf{Q}\mathbf{y}^T + J
\end{equation}

where $\mathbf{y}$ is a vector of the difference in the measured and simulated
marker coordinates and $J$ represents the average rate of metabolic expenditure
of the muscles and the sum of the square of the non-gait-critical joint torques
as defined by Wang et. al.

This formulation will require simulating the system over a time span from 1 to
10 minutes and ensuring that the difference in the marker positions and the
experimental measurements (sampled at 100 hz) are minimal. Wang et. al's model
was simulated for 10s at 2400 Hz resulting in an average optimization iteration
time of ~12s 50 core cluster. One goal of this work will be to optimize the
simulations and identification for speed so that my computations are possible
in a similar time frame as Wang et. al's.

For time optimal convergence times in the identification I plan to explore some
or all of the following:

\begin{description}
  \item[Reduced order plant] It may not be necessary use a 30 DoF model
    for adequate model fidelity. I will start with a simpler 3D model with
  \item[ODE Integration] Simbody includes several ODE integration routines that
    include explicit and implicit solvers. For example, CVODES may be able to
    be sped up by providing an analytical Jacobian for the implicit solver or
    parallelizing the numerical Jacobian computation with automatic
    differentiation. We may also be able to use or create a 2nd order implicit
    Euler method that can compute quickly with a stiff ground contact model.
  \item[Direct Collocation] Wang et. al uses an evolutionary derivative-free
    algorithm, CMA, that requires shooting. It is also possible to use a
    strictly implicit formulation of the system equations of motion paired with
    direct collocation \cite{Ackermann2010} for potentially faster convergence
    times even though the number of free parameters is larger.
  \item[Cloud Computing] The computations will be parallelized as much as
    possible with considerations for GPU based calculations, population-based
    optimization methods, the latest libraries (boost, eigen, odeint, etc) and
    run using Amazon's Web Service cloud computing services and MIT's
    \href{http://star.mit.edu/cluster/}{StarCluster} where costs can be as low
    as \$0.005 per core per hour by utilizing spot instances.
\end{description}

\section*{Relevance to Rehabilitation}

The results of these studies will be directly used at the CSU Human Motion \&
Control Lab to improve the controllers in powered prosthetics and
exo-skeletons.  There is a great deal of budding research on these devices with
particular applications in rehabilitation in mind. We will be working to
implement the identified controllers directly into the Indego exoskeleton and
CSU's leg robot with a below knee prosthetic. The Indego will be on the market
in 2015 with target marker on rehabilitation.

Also, in general, realistic control models that can be used in simulation for
walking and running will enhance a broad range of research, from rehabilitation
to all aspects of biomechanics. Just as realistic simulation of
musculo-skeletal models have transformed the way we implement rehabilitation,
realistic control models may will likely make a similar impact. For example,
real time feedback to clinicians on what parts of the human controller is and
isn't being utilized in a biomechanical, will change the game.

\section*{Contributions}

All of the products developed during, prior, and after the visiting scholarship
that are related to this proposal will be made available under permissible
licenses
(\href{http://opensource.org/licenses/BSD-2-Clause}{BSD}/\href{http://opensource.org/licenses/MIT}{MIT}
like and/or \href{http://creativecommons.org/licenses/by/4.0/}{CC-BY-4.0}) and
shared via paper, code, and data repositories on the world wide web. In
particular, these are the likely products that will be valuable to the
community:

\begin{description}
  \item[Plant Models] The source code and/or any Opensim xml descriptions for
    all of the plant models will be released to the SimTK community and be
    hosted on in a public Git repository (e.g.
    \href{http://github.com}{Github}).
  \item[Source Code] All of the source code (plant and controller models,
    optimization routines, cloud infrastructure provisioning, etc) will be
    shared via Github and structured such that all published results are
    computationally reproducible.
  \item[Data] The anonymized human subject data will be shared via
    \href{http://figshare.com}{Figshare} and/or institutional repositories (CSU
    or Stanford) and released under the CC-BY-4.0 License.
  \item[Papers] At least one paper will be published in an open access journal
    under the CC-BY-4.0 licenses along with supplementary material (source code
    and data) to reproduce the results. A preconfigured virtual machine that
    can run the computations on a cloud infrastructure will accompany the
    results.
\end{description}

\section*{My Background}

I am adept at mathematical modeling and simulation and have extensive
experience with both vehicle and biomechanical systems. I was initially trained
at UC Davis and specialized in Sports Biomechanics where modeling, simulation,
system identification played an important role in my
\href{http://moorepants.github.io/dissertation}{dissertation} work. I
co-develop and run a project, \href{http://pydy.org}{PyDy}, that provides an
open source modeling and simulation framework for complex multi-body systems.
I have used this software to develop a variety of complex vehicle and
biomechanical models which were used in the direct identification of the human
controller in the bicycle balance and control task in my dissertation. My
\href{http://github.com/moorepants}{Github profile} provides a ``resume'' of
sorts showing my open source contributions to scientific computing in vehicle
dynamics and biomechanics, with the latest being a gait analysis toolkit and
the most popular being \href{http://www.sympy.org}{SymPy}, a computer algebra
system in Python. In addition, I am proficient at several programming
languages, numerical methods, agile software development, and working on large
Open Source projects.

I am new to Opensim and Simbody due to the fact the tools were not used in the
laboratories that I've worked at thus far, and I've spent most of my time
developing our projects for simulation needs. But I've spent some time this
year working through the tutorials available for Simbody and Opensim and am now
lightly proficient. I will be spending more time with the software over the
next six months so that I can effectively make use of them in this project.

I have the skills and resources needed to complete this project and am excited
to learn from the Opemsim team and become part of that community.

\bibliographystyle{plain}
\bibliography{references}

\end{document}
